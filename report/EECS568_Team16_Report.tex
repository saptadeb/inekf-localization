%%%%%%%%%%%%%%%%%%%%%%%%%%%%%%%%%%%%%%%%%%%%%%%%%%%%%%%%%%%%%%%%%%%%%%%%%%%%%%%%
%2345678901234567890123456789012345678901234567890123456789012345678901234567890
%        1         2         3         4         5         6         7         8

\documentclass[letterpaper, 10 pt, conference]{ieeeconf}  % Comment this line out if you need a4paper

%\documentclass[a4paper, 10pt, conference]{ieeeconf}      % Use this line for a4 paper

\IEEEoverridecommandlockouts                              % This command is only needed if 
                                                          % you want to use the \thanks command

\overrideIEEEmargins                                      % Needed to meet printer requirements.

%In case you encounter the following error:
%Error 1010 The PDF file may be corrupt (unable to open PDF file) OR
%Error 1000 An error occurred while parsing a contents stream. Unable to analyze the PDF file.
%This is a known problem with pdfLaTeX conversion filter. The file cannot be opened with acrobat reader
%Please use one of the alternatives below to circumvent this error by uncommenting one or the other
%\pdfobjcompresslevel=0
%\pdfminorversion=4

% See the \addtolength command later in the file to balance the column lengths
% on the last page of the document

% The following packages can be found on http:\\www.ctan.org
%\usepackage{graphics} % for pdf, bitmapped graphics files
%\usepackage{epsfig} % for postscript graphics files
%\usepackage{mathptmx} % assumes new font selection scheme installed
%\usepackage{times} % assumes new font selection scheme installed
\usepackage{amsmath} % assumes amsmath package installed
\usepackage{amssymb}  % assumes amsmath package installed
\usepackage{amsfonts}
\usepackage{tensor}
\usepackage{mathtools}

\let\proof\relax
\let\endproof\relax
\usepackage{amsthm}

\theoremstyle{definition}
\newtheorem{definition}{Definition}
\newtheorem{theorem}{Theorem}

\title{\LARGE \bf
Invariant Extended Kalman Filtering for Robot Localization \\
using IMU and GPS}

\author{Saptadeep Debnath, Anthony Liang, Gaurav Manda, Sunbochen Tang, and Hao Zhou \\
College of Engineering, University of Michigan, Ann Arbor, MI, USA \\
\texttt{\{,zhh\}@umich.edu} 
\thanks{EECS 568 Team 16 Final Project Report}
}

\begin{document}

\maketitle
\thispagestyle{empty}
\pagestyle{empty}

%%%%%%%%%%%%%%%%%%%%%%%%%%%%%%%%%%%%%%%%%%%%%%%%%%%%%%%%%%%%%%%%%%%%%%%%%%%%%%%%
\begin{abstract}

\end{abstract}


%%%%%%%%%%%%%%%%%%%%%%%%%%%%%%%%%%%%%%%%%%%%%%%%%%%%%%%%%%%%%%%%%%%%%%%%%%%%%%%%
\section{INTRODUCTION}

\section{REVIEW OF THEORETICAL BACKGROUND AND PRELIMINARIES}

We assume a matrix Lie group \cite{c1} denoted $\cal{G}$ and its
associated Lie Algebra denoted $\mathfrak{g}$. Define
\begin{equation}
{\cal{L}}_{\mathfrak{g}}: \mathbb{R}^{\mathrm{dim} \mathfrak{g}} \rightarrow \mathfrak{g},
\end{equation}
that takes an element in the tangent space of $\cal{G}$ at the identity to its corresponding matrix representation, we can write the exponential map 
\begin{equation}
\mathrm{exp}: \mathbb{R}^{\mathrm{dim} \mathfrak{g}} \rightarrow {\cal{G}},
\end{equation}
that relates a matrix Lie group to its associated Lie algebra as 
\begin{equation}
\mathrm{exp}(\boldsymbol \xi) = \mathrm{exp}_m({\cal{L}}_{\mathfrak{g}}(\boldsymbol \xi)),
\end{equation}
where $\mathrm{exp}_m(\cdot)$ is the standard matrix exponential.

A process dynamics evolving on the Lie group is denoted by
\begin{equation}\label{dynamics}
\frac{d}{dt} \mathbf{X}_t = f_{\mathbf{u}_t}(\mathbf{X}_t),
\end{equation}
where the state $\mathbf{X}_t$ lives in the Lie group $\cal{G}$ and $\mathbf{u}_t$ is an input variable. Consider the true state trajectory $\mathbf{X}_t$ and an estimate of it $\bar{\mathbf{X}}_t$. The state estimation error is defined using right or left multiplication of $\mathbf{X}_t^{-1}$ as follows.
\begin{definition}[Left and Right Invariant Error]
\textit{The right- and
left-invariant errors between two trajectories $\mathbf{X}_t$ and $\bar{\mathbf{X}}_t$ are:
\begin{eqnarray}
\label{ri_error}
{\boldsymbol \eta}^r_t = \bar{\mathbf{X}}_t \mathbf{X}_t^{-1} = (\bar{\mathbf{X}}_t \boldsymbol \Gamma) (\mathbf{X}_t \boldsymbol \Gamma)^{-1} \text{\ \ \ (Right-Invariant)} \\
\label{li_error}
{\boldsymbol \eta}^l_t = \mathbf{X}_t^{-1} \bar{\mathbf{X}}_t = ( \boldsymbol \Gamma \bar{\mathbf{X}}_t)^{-1} (\boldsymbol \Gamma \mathbf{X}_t) \text{\ \ \ (Left-Invariant)},
\end{eqnarray}
where $\boldsymbol \Gamma \in {\cal{G}}$ is an arbitrary element of the group.
}
\end{definition}
The following two theorems are the fundamental results for deriving an Invariant extended Kalman filter and show that a wide range of nonlinear problems can lead to linear error equations provided the error variable is correctly chosen.
\begin{theorem} [Autonomous Error Dynamics] \label{th1}
\textit{A system is
group affine if the dynamics $f_{\mathbf{u}_t} (\cdot)$ satisfies: for all $t > 0$ and $\mathbf{X}_1, \mathbf{X}_2 \in {\cal{G}}$:
\begin{equation} \label{group_affine}
f_{\mathbf{u}_t} (\mathbf{X}_1 \mathbf{X}_2) = f_{\mathbf{u}_t} (\mathbf{X}_1) \mathbf{X}_2 + \mathbf{X}_1 f_{\mathbf{u}_t} (\mathbf{X}_2) - \mathbf{X}_1 f_{\mathbf{u}_t} (\mathbf{I}_d) \mathbf{X}_2,
\end{equation}
where $\mathbf{I}_d$ denotes the identity matrix. Furthermore, if this condition is satisfied, the right- and left-invariant error dynamics are trajectory independent and satisfy:
\begin{eqnarray}
\frac{d}{dt} {\boldsymbol \eta}^r_t = g_{\mathbf{u}_t}^r({\boldsymbol \eta}^r_t), \text{ where } g_{\mathbf{u}_t}^r({\boldsymbol \eta}) = f_{\mathbf{u}_t}({\boldsymbol \eta}) - {\boldsymbol \eta} f_{\mathbf{u}_t}(\mathbf{I}_d) \\
\frac{d}{dt} {\boldsymbol \eta}^l_t = g_{\mathbf{u}_t}^l({\boldsymbol \eta}^l_t), \text{ where } g_{\mathbf{u}_t}^l({\boldsymbol \eta}) = f_{\mathbf{u}_t}({\boldsymbol \eta}) - f_{\mathbf{u}_t}(\mathbf{I}_d) {\boldsymbol \eta}.
\end{eqnarray}
}
\end{theorem}

\begin{theorem} [Log-linear Property of the Error] \label{th2}
\textit{Consider the right- or left-invariant error ${\boldsymbol \eta}^i_t$ as defined by (\ref{ri_error}) and (\ref{li_error}) between two arbitrarily far trajectories of (\ref{dynamics}) and (\ref{group_affine}), the superscript $i$ denoting indifferently $l$ or $r$. Let ${\cal{L}}_\mathfrak{g}$ and $\mathrm{exp}(\cdot)$ be defined as above. Let ${\boldsymbol \xi}_0^i \in \mathbb{R}^{\mathrm{dim} \mathfrak{g}}$ be such that initially 
\begin{equation}
\mathrm{exp}(\boldsymbol \xi_0^i) =  {\boldsymbol \eta}^i_0. 
\end{equation} 
Let $\mathbf{A}^i_t$ be defined by
\begin{equation}
g_{\mathbf{u_t}}^i (\mathrm{exp}(\boldsymbol \xi)) = {\cal{L}}_{\mathfrak{g}} (\mathbf{A}^i_t \boldsymbol \xi) + O(\|\boldsymbol \xi\|^2).
\end{equation} 
If ${\boldsymbol \xi}_t^i$ is defined for $t > 0$ by the linear differential equation in $\mathbb{R}^{\mathrm{dim} \mathfrak{g}}$
\begin{equation} \label{error_dynamic}
\frac{d}{dt} {\boldsymbol \xi}_t^i = \mathbf{A}^i_t {\boldsymbol \xi}^i_t,
\end{equation}
then, we have for the true non-linear error ${\boldsymbol \eta}^i_t$, the correspondence at all times and for arbitrarily large errors
\begin{equation}
\forall t \geq 0, {\boldsymbol \eta}^i_t = \mathrm{exp}({\boldsymbol \xi}_t^i).
\end{equation}
}
\end{theorem}
This theorem states that (\ref{error_dynamic}) is not the typical Jacobian linearization along a trajectory because the left- or right-invariant error on the Lie group can be exactly recovered from its solution. This result is of major importance for the propagation (prediction) step of the InEKF.

The adjoint representation plays a key role in the theory of Lie groups and through this linear map we can capture the non-commutative structure of a Lie group.
\begin{definition}[The Adjoint Map]
\textit{Let $\cal{G}$ be a matrix Lie group with Lie algebra $\mathfrak{g}$. For any $\mathbf{X} \in {\cal{G}}$, the adjoint map, \begin{equation}
Ad_\mathbf{X}: \mathfrak{g} \rightarrow \mathfrak{g},
\end{equation}
is a linear map defined as
\begin{equation}
Ad_\mathbf{X}({\cal{L}}_\mathfrak{g}(\boldsymbol \xi)) = X {\cal{L}}_\mathfrak{g}(\boldsymbol \xi) X^{-1}.
\end{equation}
Furthermore, we denote the matrix representation of the adjoint map by $\mathrm{Ad}_\mathbf{X}$, such that
\begin{equation}
Ad_\mathbf{X}({\cal{L}}_\mathfrak{g}(\boldsymbol \xi)) = {\cal{L}}_\mathfrak{g}(\mathrm{Ad}_\mathbf{X} \boldsymbol \xi).
\end{equation}
}
\end{definition}

\section{$SE_2(3)$ CONTINUOUS LEFT-INVARIANT EKF}
In this section, we derive a Left-Invariant Extended Kalman Filter (LI-EKF) using IMU and GPS measurements. IMU biases are neglected for now to be consistent with the standard InEKF theory. Section ??? provides a method for reintroducing the bias terms.

\subsection{State Representation}
As with typical aided inertial navigation, we wish to estimate the orientation, velocity, and position of the robot (IMU) in the world frame. These states are represented by $\tensor[]{\mathbf{R}}{_{WB}}(t), \tensor[_W]{\mathbf{v}}{_{WB}}(t)$, and $\tensor[_W]{\mathbf{p}}{_{WB}} (t)$ respectively. The above collection of state variables forms the group of double direct spatial isometrie $SE_2(3)$. Specifically, $\mathbf{X}_t \in SE_2(3)$ can be represented by the following matrix:
\begin{equation}
\mathbf{X}_t \triangleq \left[ \begin{array}{ccc}
\tensor[]{\mathbf{R}}{_{WB}}(t) & \tensor[_W]{\mathbf{v}}{_{WB}}(t) & \tensor[_W]{\mathbf{p}}{_{WB}} (t) \\
\mathbf{0}_{1 \times 3} & 1 & 0 \\
\mathbf{0}_{1 \times 3} & 0 & 1
\end{array} \right].
\end{equation}
We introduce the following shorthand notation:
\begin{eqnarray}
\mathbf{X}_t \triangleq \left[ \begin{array}{ccc}
\mathbf{R}_t & \mathbf{v}_t & \mathbf{p}_t \\
\mathbf{0}_{1 \times 3} & 1 & 0 \\
\mathbf{0}_{1 \times 3} & 0 & 1
\end{array} \right], \\
\mathbf{u}_t \triangleq \left[ \begin{array}{c}
\tensor[_B]{\tilde{\boldsymbol \omega}}{_{WB}}(t) \\
\tensor[_B]{\tilde{\mathbf{a}}}{_{WB}}(t)
\end{array} \right] 
\triangleq \left[ \begin{array}{c}
{\boldsymbol \omega}_t \\
\mathbf{a}_t \\
\end{array} \right],
\end{eqnarray}
where the input $\mathbf{u}_t$ is formed from the angular velocity and linear acceleration measurements coming from the IMU. It is important to note that these measurements are taken in the
body (or IMU) frame. State $\mathbf{X}_t$ has a total of 9 degrees of freedom (3 for rotation, 3 for velocity, and 3 for position), indicating the dimension of the associated Lie algebra is also 9. The Lie algebra of $SE_2(3)$, denoted by $\mathfrak{se}_2(3)$, is a 5 dimensional square matrix. Defining a map, ${\cal{L}}_\mathfrak{g}: \mathbb{R}^9 \rightarrow \mathfrak{g}$, that maps a vector $\boldsymbol \xi \in \mathbb{R}^9$ to the corresponding element of the Lie algebram, we can write it as:
\begin{equation}
{\cal{L}}_\mathfrak{g}(\boldsymbol \xi) = 
{\cal{L}}_\mathfrak{g} \left( \left[ \begin{array}{c}
{\boldsymbol \xi}^R \\
{\boldsymbol \xi}^v \\
{\boldsymbol \xi}^p
\end{array} \right] \right) =
\left[ \begin{array}{ccc}
\left({\boldsymbol \xi}^R\right)^{\wedge} & {\boldsymbol \xi}^v & {\boldsymbol \xi}^p \\
\mathbf{0}_{1 \times 3} & 0 & 0 \\
\mathbf{0}_{1 \times 3} & 0 & 0
\end{array} \right],
\end{equation} 
where $(\cdot)^{\wedge}$ denotes a $3 \times 3$ skew-symmetric matrix.

The exponential mapping is given by the formula:
\begin{equation}
\mathrm{exp}(\boldsymbol \xi) = \mathbf{I}_5 + \mathbf{S} + \frac{1-\cos\|\boldsymbol \xi\|}{\|\boldsymbol \xi\|^2}\mathbf{S}^2 + \frac{\|\boldsymbol \xi\|-\sin \|\boldsymbol \xi\|}{\|\boldsymbol \xi\|^3}\mathbf{S}^3,
\end{equation}
where $\mathbf{S} = {\cal{L}}_\mathfrak{g}(\boldsymbol \xi)$.

Matrix representation of the adjoint operator is given by:
\begin{equation}
\mathrm{Ad}_{\mathbf{X}_t} = \left[ \begin{array}{ccc}
\mathbf{R}_t & 0 & 0 \\
\mathbf{v}_t \times \mathbf{R}_t & \mathbf{R}_t & 0 \\
\mathbf{p}_t \times \mathbf{R}_t & 0 & \mathbf{R}_t
\end{array} \right].
\end{equation}

\subsection{Continuous System Dynamics}
The IMU measurements are modeled as being corrupted by additive white Gaussian noise, per
\begin{eqnarray}
\tilde{\boldsymbol \omega}_t = {\boldsymbol \omega}_t + \mathbf{w}^g_t, \mathbf{w}^g_t \sim {\cal{GP}}(\mathbf{0}_{3 \times 1}, \Sigma^g \delta(t-t')) \\
\tilde{\mathbf{a}}_t = {\mathbf{a}}_t + \mathbf{w}^a_t, \mathbf{w}^a_t \sim {\cal{GP}}(\mathbf{0}_{3 \times 1}, \Sigma^a \delta(t-t')).
\end{eqnarray}
The IMU dynamics can be written as:
\begin{eqnarray}
\dot{\mathbf{R}}_t & = & \mathbf{R}_t (\tilde{\boldsymbol \omega}_t - \mathbf{w}^g_t)^{\wedge} \\
\dot{\mathbf{v}}_t & = & \mathbf{R}_t (\tilde{\mathbf{a}}_t - \mathbf{w}_t^a) + \mathbf{g} \\
\dot{\mathbf{p}}_t & = & \mathbf{v}_t,
\end{eqnarray}
where $\mathbf{g}$ is the gravity vector.

In matrix form, the dynamics can be expressed as
\begin{multline} \label{noisy_system}
\frac{d}{dt}\mathbf{X}_t = \left[ \begin{array}{ccc}
\mathbf{R}_t \tilde{\boldsymbol \omega}_t^{\wedge} & \mathbf{R}_t \tilde{\mathbf{a}}_t + \mathbf{g} & \mathbf{v}_t \\
\mathbf{0}_{1 \times 3} & 0 & 0 \\
\mathbf{0}_{1 \times 3} & 0 & 0
\end{array} \right] \\
- \left[ \begin{array}{ccc}
\mathbf{R}_t & \mathbf{v}_t & \mathbf{p}_t \\
\mathbf{0}_{1 \times 3} & 0 & 0 \\
\mathbf{0}_{1 \times 3} & 0 & 0
\end{array} \right] 
\left[ \begin{array}{ccc}
\left( \mathbf{w}_t^g \right)^{\wedge} & \mathbf{w}_t^a & \mathbf{0}_{3 \times 1} \\
\mathbf{0}_{1 \times 3} & 0 & 0 \\
\mathbf{0}_{1 \times 3} & 0 & 0
\end{array} \right] \\
\triangleq f_{\mathbf{u}_t}(\mathbf{X}_t) - \mathbf{X}_t {\cal{L}}_{\mathfrak{g}}(\mathbf{w}_t), \hspace{3.2cm}                 
\end{multline}
where $\mathbf{w}_t = \left[ \left(\mathbf{w}_t^g\right)^T, \left(\mathbf{w}_t^a\right)^T , \mathbf{0}_{1 \times 3} \right]^T$. The deterministic system dynamics, $f_{\mathbf{u}_t}(\cdot)$, can be shown to satisfy the group affine property (\ref{group_affine}). Therefore, following Theorem \ref{th1}, the left- and right-invariant error dynamics will evolve independently of the system's state. Elementary computations based on the results of Theorem \ref{th1} show that for the noisy model (\ref{noisy_system}) we have
\begin{multline} \label{noisy_error_G}
\frac{d}{dt} {\boldsymbol \eta}^l_t = f_{\mathbf{u}_t}({\boldsymbol \eta}^l_t) - f_{\mathbf{u}_t}(\mathbf{I}_d) {\boldsymbol \eta}^l_t + {\cal{L}}_{\mathfrak{g}}(\mathbf{w}_t) {\boldsymbol \eta}^l_t \\
= g_{\textbf{u}_t}^l({\boldsymbol \eta}^l_t) + {\cal{L}}_{\mathfrak{g}}(\mathbf{w}_t) {\boldsymbol \eta}^l_t, \hspace{3.1cm}
\end{multline}
where the second term arises from the additive noise. The
derivation follows the results in [?????] and is not repeated here.

Theorem \ref{th2} furthermore specifies that the invariant error satisfies a log-linear property. Namely, if $\mathbf{A}_t$ is defined by 
\begin{equation}
g_{\mathbf{u_t}}^l (\mathrm{exp}(\boldsymbol \xi)) = {\cal{L}}_{\mathfrak{g}} (\mathbf{A}_t \boldsymbol \xi) + O(\|\boldsymbol \xi\|^2), 
\end{equation}
then the log of the invariant error, $\boldsymbol \xi \in \mathbb{R}^{\mathrm{dim} {\mathfrak{g}}}$, satisfies the linear system
\begin{equation} \label{noisy_error_dimg}
\frac{d}{dt} {\boldsymbol \xi}_t = \mathbf{A}_t {\boldsymbol \xi}_t +  \mathbf{w}_t.
\end{equation}
We provide a quick proof of (\ref{noisy_error_dimg}) below, from (\ref{noisy_error_G}),
\begin{multline}
\frac{d}{dt} {\boldsymbol \eta}_t = \frac{d}{dt} \mathrm{exp}({\boldsymbol \xi}_t) \approx \frac{d}{dt} \left(\mathbf{I}_d + {\cal{L}}_{\mathfrak{g}} ({\boldsymbol \xi}_t) \right) = \frac{d}{dt} \left({\cal{L}}_{\mathfrak{g}} ({\boldsymbol \xi}_t) \right) \\
= g_{\textbf{u}_t}^l \left( \mathrm{exp}({\boldsymbol \xi}_t) \right) + {\cal{L}}_{\mathfrak{g}}(\mathbf{w}_t) \mathrm{exp}({\boldsymbol \xi}_t) \hspace{2.3cm} \\
\approx {\cal{L}}_{\mathfrak{g}} (\mathbf{A}_t {\boldsymbol \xi}_t) + O(\|{\boldsymbol \xi}_t\|^2) + {\cal{L}}_{\mathfrak{g}}(\mathbf{w}_t) \left(\mathbf{I}_d + {\cal{L}}_{\mathfrak{g}} ({\boldsymbol \xi}_t) \right) \\
= {\cal{L}}_{\mathfrak{g}} \left(\mathbf{A}_t {\boldsymbol \xi}_t + \mathbf{w}_t \right) + O(\|{\boldsymbol \xi}_t\|^2) + O(\|\mathbf{w}_t \| \|{\boldsymbol \xi}_t\|), \\
\implies \frac{d}{dt} \left({\cal{L}}_{\mathfrak{g}} ({\boldsymbol \xi}_t) \right) = {\cal{L}}_{\mathfrak{g}} \left(\mathbf{A}_t {\boldsymbol \xi}_t + \mathbf{w}_t \right), \\
\implies \frac{d}{dt} {\boldsymbol \xi}_t = \mathbf{A}_t {\boldsymbol \xi}_t +  \mathbf{w}_t. \hspace{2.4cm}
\end{multline}
To compute the matrix $\mathbf{A}_t$, we linearize the invariant error dynamics, $g_{\mathbf{u}_t}^l(\cdot)$, using the first order approximation
\begin{equation}
{\boldsymbol \eta}_t^l = \mathrm{exp}({\boldsymbol \xi}_t) \approx \mathbf{I}_d + {\cal{L}}_{\mathfrak{g}} ({\boldsymbol \xi}_t),
\end{equation}
to yield
\begin{multline}
g_{\mathbf{u}_t}^l ( {\boldsymbol \eta}_t^l ) = g_{\mathbf{u}_t}^l \left( \mathbf{I}_d + {\cal{L}}_{\mathfrak{g}} ({\boldsymbol \xi}_t) \right) \\
= f_{\mathbf{u}_t} \left( \mathbf{I}_d + {\cal{L}}_{\mathfrak{g}} ({\boldsymbol \xi}_t) \right) - f_{\mathbf{u}_t}(\mathbf{I}_d) \left( \mathbf{I}_d + {\cal{L}}_{\mathfrak{g}} ({\boldsymbol \xi}_t) \right) \\
= \left[ \begin{array}{ccc}
\left( \mathbf{I} + \left( {\boldsymbol \xi}_t^R \right)^{\wedge} \right) \tilde{\boldsymbol \omega}_t^{\wedge} & \left( \mathbf{I} + \left( {\boldsymbol \xi}_t^R \right)^{\wedge} \right) \tilde{\mathbf{a}}_t + \mathbf{g} & {\boldsymbol \xi}_t^v \\
\mathbf{0}_{1 \times 3} & 0 & 0 \\
\mathbf{0}_{1 \times 3} & 0 & 0
\end{array} \right] \\
- \left[ \begin{array}{ccc}
\tilde{\boldsymbol \omega}_t^{\wedge} & \tilde{\mathbf{a}}_t + \mathbf{g} & \mathbf{0}_{3,1} \\
\mathbf{0}_{1 \times 3} & 0 & 0 \\
\mathbf{0}_{1 \times 3} & 0 & 0
\end{array} \right] 
\left[ \begin{array}{ccc}
\mathbf{I} + \left( {\boldsymbol \xi}_t^R \right)^{\wedge} &  {\boldsymbol \xi}_t^v & {\boldsymbol \xi}_t^p \\
\mathbf{0}_{1 \times 3} & 1 & 0 \\
\mathbf{0}_{1 \times 3} & 0 & 1
\end{array} \right] \\
= \left[ \begin{array}{ccc}
G_{11} & G_{12} & G_{13} \\
\mathbf{0}_{1 \times 3} & 0 & 0 \\
\mathbf{0}_{1 \times 3} & 0 & 0
\end{array} \right], \hspace{4.1cm} \\
\text{where } \mathbf{I} = \mathbf{I}_{3 \times 3}, \hspace{5.8cm} \\
G_{11} = \left( {\boldsymbol \xi}_t^R \right)^{\wedge} \tilde{\boldsymbol \omega}_t^{\wedge} - \tilde{\boldsymbol \omega}_t^{\wedge} \left( {\boldsymbol \xi}_t^R \right)^{\wedge} = \left( -\tilde{\boldsymbol \omega}_t^{\wedge} {\boldsymbol \xi}_t^R \right)^{\wedge}, \\
G_{12} = \left( {\boldsymbol \xi}_t^R \right)^{\wedge} \tilde{\mathbf{a}}_t -\tilde{\boldsymbol \omega}_t^{\wedge} {\boldsymbol \xi}_t^v, \hspace{3.2cm} \\
G_{13} = {\boldsymbol \xi}_t^v - \tilde{\boldsymbol \omega}_t^{\wedge} {\boldsymbol \xi}_t^p. \hspace{4.15cm}
\end{multline}
We can further write $g_{\mathbf{u}_t}^l ( \cdot )$ as
\begin{multline} \label{Axi}
g_{\mathbf{u}_t}^l ( {\boldsymbol \eta}_t^l ) = 
\left[ \begin{array}{ccc}
G_{11} & G_{12} & G_{13} \\
\mathbf{0}_{1 \times 3} & 0 & 0 \\
\mathbf{0}_{1 \times 3} & 0 & 0
\end{array} \right] \\
= {\cal{L}}_{\mathfrak{g}} \left( \left[ \begin{array}{c}
-\tilde{\boldsymbol \omega}_t^{\wedge} {\boldsymbol \xi}_t^R \\
- \tilde{\mathbf{a}}_t^{\wedge} {\boldsymbol \xi}_t^R -\tilde{\boldsymbol \omega}_t^{\wedge} {\boldsymbol \xi}_t^v \\
{\boldsymbol \xi}_t^v - \tilde{\boldsymbol \omega}_t^{\wedge} {\boldsymbol \xi}_t^p
\end{array} \right] \right) \\
= {\cal{L}}_{\mathfrak{g}} \left( \left[ \begin{array}{ccc}
-\tilde{\boldsymbol \omega}_t^{\wedge} & \mathbf{0}_{3 \times 3} & \mathbf{0}_{3 \times 3} \\
- \tilde{\mathbf{a}}_t^{\wedge} & -\tilde{\boldsymbol \omega}_t^{\wedge} & \mathbf{0}_{3 \times 3} \\
\mathbf{0}_{3 \times 3} & \mathbf{I}_{3 \times 3} & - \tilde{\boldsymbol \omega}_t^{\wedge}
\end{array} \right] 
\left[ \begin{array}{c}
{\boldsymbol \xi}_t^R \\
{\boldsymbol \xi}_t^v \\
{\boldsymbol \xi}_t^p
\end{array} \right] 
\right).
\end{multline}
With the above, we can express the prediction step of the
LI-EKF. The state estimate, $\hat{\mathbf{X}}_t$, is propagated though the deterministic system dynamics, while the covariance matrix, $\mathbf{P}_t$, is computed using the Riccati equation, namely,
\begin{equation}
\frac{d}{dt} \hat{\mathbf{X}}_t = f_{\mathbf{u}_t} (\hat{\mathbf{X}}_t), \frac{d}{dt} \mathbf{P}_t = \mathbf{A}_t \mathbf{P}_t + \mathbf{P}_t \mathbf{A}_t + \hat{\mathbf{Q}}_t,
\end{equation}
where the matrices $\mathbf{A}_t$ and $\hat{\mathbf{Q}}_t$ are obtained from (\ref{Axi}) and (\ref{noisy_error_dimg}),
\begin{equation}
\mathbf{A}_t = \left[ \begin{array}{ccc}
-\tilde{\boldsymbol \omega}_t^{\wedge} & \mathbf{0}_{3 \times 3} & \mathbf{0}_{3 \times 3} \\
- \tilde{\mathbf{a}}_t^{\wedge} & -\tilde{\boldsymbol \omega}_t^{\wedge} & \mathbf{0}_{3 \times 3} \\
\mathbf{0}_{3 \times 3} & \mathbf{I}_{3 \times 3} & - \tilde{\boldsymbol \omega}_t^{\wedge}
\end{array} \right],
\hat{\mathbf{Q}}_t = \mathrm{Cov}[\mathbf{w}_t].
\end{equation}

\subsection{Discretization}
The continuous dynamics can be discretized by assuming a zero-order hold on the inputs and performing Euler integration from $t_{k-1}$ to $t_k$. The discrete dynamics for the individual state elements becomes:
\begin{eqnarray}
\hat{\mathbf{R}}_{t_k} & = & \mathbf{R}_{t_{k-1}} \mathrm{exp}(\tilde{\boldsymbol \omega}_t \Delta t) \\
\hat{\mathbf{v}}_{t_k} & = & \mathbf{v}_{t_{k-1}} + (\mathbf{R}_{t_{k-1}} \tilde{\mathbf{a}}_t + \mathbf{g}) \Delta t \\
\hat{\mathbf{p}}_{t_k} & = & \mathbf{p}_{t_{k-1}} + \mathbf{v}_{t_{k-1}} \Delta t + \frac{1}{2} (\mathbf{R}_{t_{k-1}} \tilde{\mathbf{a}}_t + \mathbf{g}) \Delta t^2,
\end{eqnarray}
where $\Delta t = t_k - t_{k-1}$ and $\mathrm{exp}(\cdot)$ is the exponential map for $SO(3)$. A first-order approximation can be used to simplify integration of the Riccati equation, resulting in the following discrete-time covariance propagation equation,
\begin{equation}
\mathbf{P}_k = {\boldsymbol \Phi} \mathbf{P}_{k-1} {\boldsymbol \Phi}^T + \hat{\mathbf{Q}}_{k-1},
\end{equation}
where
\begin{eqnarray}
{\boldsymbol \Phi} & = & \mathrm{exp}_m(\mathbf{A}_t \Delta t) \\
\hat{\mathbf{Q}}_{k-1} & \approx & {\boldsymbol \Phi} \hat{\mathbf{Q}}_t {\boldsymbol \Phi}^T \Delta t.
\end{eqnarray}

\subsection{Left-invariant Measurement Model}
The GPS measurement model corresponds to the left-invariant observation form: $\mathbf{Y}_{t_k} = \hat{\mathbf{X}}_{t_k} \mathbf{b} + \mathbf{V}_{t_k}$,
\begin{equation}
\left[ \begin{array}{c}
\mathbf{y}_{t_k} \\
0 \\
1 \\
\end{array} \right] =
\left[ \begin{array}{ccc}
\hat{\mathbf{R}}_{t_k} & \hat{\mathbf{v}}_{t_k} & \hat{\mathbf{p}}_{t_k} \\
\mathbf{0}_{1 \times 3} & 1 & 0 \\
\mathbf{0}_{1 \times 3} & 0 & 1 \\
\end{array} \right] 
\left[ \begin{array}{c}
\mathbf{0}_{3 \times 1} \\
0 \\
1
\end{array} \right] + 
\left[ \begin{array}{c}
\mathbf{v}_{t_k} \\
0 \\
0
\end{array} \right]
\end{equation}
Therefore, the innovation depends solely on the invariant error
and the update equations take the form
\begin{eqnarray}
\hat{\mathbf{X}}_{t_k}^+ = \hat{\mathbf{X}}_{t_k} \mathrm{exp} \left( \mathbf{L}_{t_k} \left( \hat{\mathbf{X}}_{t_k}^{-1} \mathbf{Y}_{t_k} - \mathbf{b} \right) \right) \\
{\boldsymbol \eta}_{t_k}^{l+} = {\boldsymbol \eta}_{t_k}^l \mathrm{exp} \left( \mathbf{L}_{t_k} \left( \left({\boldsymbol \eta}_{t_k}^l \right)^{-1} \mathbf{b} - \mathbf{b} + \hat{\mathbf{X}}_{t_k}^{-1} \mathbf{V}_{t_k} \right) \right)
\end{eqnarray}
where $\mathrm{exp}(\cdot)$ is the exponential map corresponding to the state matrix Lie group, $\mathbf{L}_{t_k}$ is a gain matrix to be defined later. Linearizing the left-invariant error ${\boldsymbol \eta}_{t_k}^l$ and neglecting the higher order terms, we get
\begin{equation}
{{\boldsymbol \xi}_{t_k}^l}^+ = {\boldsymbol \xi}_{t_k}^l + \mathbf{L}_{t_k}(-{\cal{L}}_{\mathfrak{g}} ({\boldsymbol \xi}_{t_k}^l) \mathbf{b} + \hat{\mathbf{X}}_{t_k}^{-1} \mathbf{V}_{t_k})
\end{equation}
Define the measurement Jacobian, $\mathbf{H}$, such that
\begin{multline}
\mathbf{H} {\boldsymbol \xi} = {\cal{L}}_{\mathfrak{g}} ({\boldsymbol \xi}) \mathbf{b} = 
\left[ \begin{array}{ccc}
\left({\boldsymbol \xi}^R\right)^{\wedge} & {\boldsymbol \xi}^v & {\boldsymbol \xi}^p \\
\mathbf{0}_{1 \times 3} & 0 & 0 \\
\mathbf{0}_{1 \times 3} & 0 & 0
\end{array} \right]
\left[ \begin{array}{c}
\mathbf{0}_{3 \times 1} \\
0 \\
1
\end{array} \right] \\
= \left[ \begin{array}{c}
{\boldsymbol \xi}^p \\
0 \\
0
\end{array} \right] =
\left[ \begin{array}{ccc}
\mathbf{0}_{3 \times 3} & \mathbf{0}_{3 \times 3} & \mathbf{I}_{3 \times 3} \\
\mathbf{0}_{1 \times 3} & 0 & 0 \\
\mathbf{0}_{1 \times 3} & 0 & 0
\end{array} \right]
\left[ \begin{array}{c}
{\boldsymbol \xi}^R \\
{\boldsymbol \xi}^v \\
{\boldsymbol \xi}^p
\end{array} \right],
\end{multline}
and in its reduced form,
\begin{equation}
\mathbf{H} = \left[ \begin{array}{ccc}
\mathbf{0}_{3 \times 3} & \mathbf{0}_{3 \times 3} & \mathbf{I}_{3 \times 3}
\end{array} \right].
\end{equation}
We can further write the linearized error as
\begin{eqnarray}
{\boldsymbol \xi}_{t_k}^{l+} & = & {\boldsymbol \xi}_{t_k}^l - \mathbf{L}_{t_k} \mathbf{H} {\boldsymbol \xi}_{t_k}^l + \mathbf{L}_{t_k} \hat{\mathbf{X}}_{t_k}^{-1} \mathbf{V}_{t_k} \\
& = & (\mathbf{I} - \mathbf{L}_{t_k} \mathbf{H}) {\boldsymbol \xi}_{t_k}^l + \mathbf{L}_{t_k} \hat{\mathbf{X}}_{t_k}^{-1} \mathbf{V}_{t_k} 
\end{eqnarray}
Finally, we can write down the full state and covariance update equations of the LI-EKF using the derived linear update equation and the theory of Kalman filtering as
\begin{eqnarray}
\hat{\mathbf{X}}_{t_k}^+ = \hat{\mathbf{X}}_{t_k} \mathrm{exp} \left( \mathbf{L}_{t_k} \left( \hat{\mathbf{X}}_{t_k}^{-1} \mathbf{Y}_{t_k} - \mathbf{b} \right) \right) \\
\mathbf{P}_{t_k}^+ = (\mathbf{I} - \mathbf{L}_{t_k} \mathbf{H}) \mathbf{P}_{t_k} (\mathbf{I} - \mathbf{L}_{t_k} \mathbf{H})^T + \mathbf{L}_{t_k} \hat{\mathbf{N}}_{t_k} \mathbf{L}_{t_k}^T,
\end{eqnarray}
where
\begin{eqnarray}
\hat{\mathbf{N}}_{t_k} = \hat{\mathbf{X}}_{t_k}^{-1} \mathrm{Cov}[\mathbf{V}_{t_k}]\hat{\mathbf{X}}_{t_k}^{-T}, \\
\mathbf{L}_{t_k} = \mathbf{P}_{t_k} \mathbf{H}^T \mathbf{S}^{-1}, \\
\mathbf{S} = \mathbf{H} \mathbf{P}_{t_k} \mathbf{H}^T + \hat{\mathbf{N}}_{t_k}.
\end{eqnarray}

\section{INCLUDING IMU BIASES}

\section{RESULTS}

\section{CONCLUSIONS}

A conclusion section is not required. Although a conclusion may review the main points of the paper, do not replicate the abstract as the conclusion. A conclusion might elaborate on the importance of the work or suggest applications and extensions. 

\addtolength{\textheight}{-12cm}   % This command serves to balance the column lengths
                                  % on the last page of the document manually. It shortens
                                  % the textheight of the last page by a suitable amount.
                                  % This command does not take effect until the next page
                                  % so it should come on the page before the last. Make
                                  % sure that you do not shorten the textheight too much.

%%%%%%%%%%%%%%%%%%%%%%%%%%%%%%%%%%%%%%%%%%%%%%%%%%%%%%%%%%%%%%%%%%%%%%%%%%%%%%%%



%%%%%%%%%%%%%%%%%%%%%%%%%%%%%%%%%%%%%%%%%%%%%%%%%%%%%%%%%%%%%%%%%%%%%%%%%%%%%%%%



%%%%%%%%%%%%%%%%%%%%%%%%%%%%%%%%%%%%%%%%%%%%%%%%%%%%%%%%%%%%%%%%%%%%%%%%%%%%%%%%
\section*{APPENDIX}

Appendixes should appear before the acknowledgment.

\section*{ACKNOWLEDGMENT}

The preferred spelling of the word �acknowledgment� in America is without an �e� after the �g�. Avoid the stilted expression, �One of us (R. B. G.) thanks . . .�  Instead, try �R. B. G. thanks�. Put sponsor acknowledgments in the unnumbered footnote on the first page.



%%%%%%%%%%%%%%%%%%%%%%%%%%%%%%%%%%%%%%%%%%%%%%%%%%%%%%%%%%%%%%%%%%%%%%%%%%%%%%%%

References are important to the reader; therefore, each citation must be complete and correct. If at all possible, references should be commonly available publications.



\begin{thebibliography}{99}

\bibitem{c1} Brian Hall. \textit{Lie groups, Lie algebras, and representations: an elementary introduction}, volume 222. Springer, 2015.
\bibitem{c1} Axel Barrau and Silv�re Bonnabel. \textit{The invariant extended Kalman filter as a stable observer}. IEEE Transactions on Automatic Control, 62(4):1797-1812, 2017.


\end{thebibliography}




\end{document}
